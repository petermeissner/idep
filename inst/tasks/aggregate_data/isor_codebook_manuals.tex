\documentclass[10pt,]{article}
\usepackage{lmodern}
\usepackage{amssymb,amsmath}
\usepackage{ifxetex,ifluatex}
\usepackage{fixltx2e} % provides \textsubscript
\ifnum 0\ifxetex 1\fi\ifluatex 1\fi=0 % if pdftex
  \usepackage[T1]{fontenc}
  \usepackage[utf8]{inputenc}
\else % if luatex or xelatex
  \ifxetex
    \usepackage{mathspec}
    \usepackage{xltxtra,xunicode}
  \else
    \usepackage{fontspec}
  \fi
  \defaultfontfeatures{Mapping=tex-text,Scale=MatchLowercase}
  \newcommand{\euro}{€}
\fi
% use upquote if available, for straight quotes in verbatim environments
\IfFileExists{upquote.sty}{\usepackage{upquote}}{}
% use microtype if available
\IfFileExists{microtype.sty}{%
\usepackage{microtype}
\UseMicrotypeSet[protrusion]{basicmath} % disable protrusion for tt fonts
}{}
\usepackage[margin=1in]{geometry}
\ifxetex
  \usepackage[setpagesize=false, % page size defined by xetex
              unicode=false, % unicode breaks when used with xetex
              xetex]{hyperref}
\else
  \usepackage[unicode=true]{hyperref}
\fi
\hypersetup{breaklinks=true,
            bookmarks=true,
            pdfauthor={pm/bm},
            pdftitle={ISOR Coding Procedures},
            colorlinks=true,
            citecolor=blue,
            urlcolor=blue,
            linkcolor=magenta,
            pdfborder={0 0 0}}
\urlstyle{same}  % don't use monospace font for urls
\usepackage{color}
\usepackage{fancyvrb}
\newcommand{\VerbBar}{|}
\newcommand{\VERB}{\Verb[commandchars=\\\{\}]}
\DefineVerbatimEnvironment{Highlighting}{Verbatim}{commandchars=\\\{\}}
% Add ',fontsize=\small' for more characters per line
\usepackage{framed}
\definecolor{shadecolor}{RGB}{248,248,248}
\newenvironment{Shaded}{\begin{snugshade}}{\end{snugshade}}
\newcommand{\KeywordTok}[1]{\textcolor[rgb]{0.13,0.29,0.53}{\textbf{{#1}}}}
\newcommand{\DataTypeTok}[1]{\textcolor[rgb]{0.13,0.29,0.53}{{#1}}}
\newcommand{\DecValTok}[1]{\textcolor[rgb]{0.00,0.00,0.81}{{#1}}}
\newcommand{\BaseNTok}[1]{\textcolor[rgb]{0.00,0.00,0.81}{{#1}}}
\newcommand{\FloatTok}[1]{\textcolor[rgb]{0.00,0.00,0.81}{{#1}}}
\newcommand{\ConstantTok}[1]{\textcolor[rgb]{0.00,0.00,0.00}{{#1}}}
\newcommand{\CharTok}[1]{\textcolor[rgb]{0.31,0.60,0.02}{{#1}}}
\newcommand{\SpecialCharTok}[1]{\textcolor[rgb]{0.00,0.00,0.00}{{#1}}}
\newcommand{\StringTok}[1]{\textcolor[rgb]{0.31,0.60,0.02}{{#1}}}
\newcommand{\VerbatimStringTok}[1]{\textcolor[rgb]{0.31,0.60,0.02}{{#1}}}
\newcommand{\SpecialStringTok}[1]{\textcolor[rgb]{0.31,0.60,0.02}{{#1}}}
\newcommand{\ImportTok}[1]{{#1}}
\newcommand{\CommentTok}[1]{\textcolor[rgb]{0.56,0.35,0.01}{\textit{{#1}}}}
\newcommand{\DocumentationTok}[1]{\textcolor[rgb]{0.56,0.35,0.01}{\textbf{\textit{{#1}}}}}
\newcommand{\AnnotationTok}[1]{\textcolor[rgb]{0.56,0.35,0.01}{\textbf{\textit{{#1}}}}}
\newcommand{\CommentVarTok}[1]{\textcolor[rgb]{0.56,0.35,0.01}{\textbf{\textit{{#1}}}}}
\newcommand{\OtherTok}[1]{\textcolor[rgb]{0.56,0.35,0.01}{{#1}}}
\newcommand{\FunctionTok}[1]{\textcolor[rgb]{0.00,0.00,0.00}{{#1}}}
\newcommand{\VariableTok}[1]{\textcolor[rgb]{0.00,0.00,0.00}{{#1}}}
\newcommand{\ControlFlowTok}[1]{\textcolor[rgb]{0.13,0.29,0.53}{\textbf{{#1}}}}
\newcommand{\OperatorTok}[1]{\textcolor[rgb]{0.81,0.36,0.00}{\textbf{{#1}}}}
\newcommand{\BuiltInTok}[1]{{#1}}
\newcommand{\ExtensionTok}[1]{{#1}}
\newcommand{\PreprocessorTok}[1]{\textcolor[rgb]{0.56,0.35,0.01}{\textit{{#1}}}}
\newcommand{\AttributeTok}[1]{\textcolor[rgb]{0.77,0.63,0.00}{{#1}}}
\newcommand{\RegionMarkerTok}[1]{{#1}}
\newcommand{\InformationTok}[1]{\textcolor[rgb]{0.56,0.35,0.01}{\textbf{\textit{{#1}}}}}
\newcommand{\WarningTok}[1]{\textcolor[rgb]{0.56,0.35,0.01}{\textbf{\textit{{#1}}}}}
\newcommand{\AlertTok}[1]{\textcolor[rgb]{0.94,0.16,0.16}{{#1}}}
\newcommand{\ErrorTok}[1]{\textcolor[rgb]{0.64,0.00,0.00}{\textbf{{#1}}}}
\newcommand{\NormalTok}[1]{{#1}}
\usepackage{longtable,booktabs}
\setlength{\parindent}{0pt}
\setlength{\parskip}{6pt plus 2pt minus 1pt}
\setlength{\emergencystretch}{3em}  % prevent overfull lines
\providecommand{\tightlist}{%
  \setlength{\itemsep}{0pt}\setlength{\parskip}{0pt}}
\setcounter{secnumdepth}{5}

%%% Use protect on footnotes to avoid problems with footnotes in titles
\let\rmarkdownfootnote\footnote%
\def\footnote{\protect\rmarkdownfootnote}

%%% Change title format to be more compact
\usepackage{titling}

% Create subtitle command for use in maketitle
\newcommand{\subtitle}[1]{
  \posttitle{
    \begin{center}\large#1\end{center}
    }
}

\setlength{\droptitle}{-2em}
  \title{ISOR Coding Procedures}
  \pretitle{\vspace{\droptitle}\centering\huge}
  \posttitle{\par}
  \author{pm/bm}
  \preauthor{\centering\large\emph}
  \postauthor{\par}
  \predate{\centering\large\emph}
  \postdate{\par}
  \date{2015-12-10 15:56:48}

\usepackage{graphicx}

% Redefines (sub)paragraphs to behave more like sections
\ifx\paragraph\undefined\else
\let\oldparagraph\paragraph
\renewcommand{\paragraph}[1]{\oldparagraph{#1}\mbox{}}
\fi
\ifx\subparagraph\undefined\else
\let\oldsubparagraph\subparagraph
\renewcommand{\subparagraph}[1]{\oldsubparagraph{#1}\mbox{}}
\fi

\begin{document}
\maketitle

{
\hypersetup{linkcolor=black}
\setcounter{tocdepth}{2}
\tableofcontents
}
\makeatletter
\newcommand{\justified}{%
  \rightskip\z@skip%
  \leftskip\z@skip} \makeatother

\newpage

\section{Introduction}\label{introduction}

To collect data on reforms to Standing Orders many steps had to be taken
and many hands had to help. The basic idea was that two proceeding
versions of the same text can be compared by putting them side by side
and going through each (sub)-paragraph.

\begin{longtable}[c]{@{}rlrl@{}}
\toprule
lnr1 & old & lnr2 & new\tabularnewline
\midrule
\endhead
1 & Three kind mice, see how they run! & 1 & Three blind mice, see how
they run!\tabularnewline
2 & They all ran after the farmer's wife, & 2 & They all ran after the
farmer's wife,\tabularnewline
3 & Who cut off their tails with the carving knife, & 3 & they took out
some cheese,\tabularnewline
4 & Did you ever see such a thing in your life? & 4 & and they cut her a
slice,\tabularnewline
5 & As three blind mice. & 5 & Did you ever see such a sight in your
life\tabularnewline
6 & End & 6 & as three kind mice?\tabularnewline
\bottomrule
\end{longtable}

Some parts might have changed, others might not have changed but were
put at a different locations. Those parts that have been changed might
have been deleted, modified or inserted.

\begin{longtable}[c]{@{}rlrlrr@{}}
\toprule
lnr1 & old & lnr2 & new & bowdist & type\tabularnewline
\midrule
\endhead
1 & Three kind mice, see how th \ldots{} & 1 & Three blind mice, see how
t \ldots{} & 2 & mod\tabularnewline
2 & They all ran after the farm \ldots{} & 2 & They all ran after the
farm \ldots{} & 0 & equal\tabularnewline
3 & Who cut off their tails wit \ldots{} & 4 & and they cut her a slice,
\ldots{} & 13 & mod\tabularnewline
& & 3 & they took out some cheese, \ldots{} & 5 & ins\tabularnewline
4 & Did you ever see such a thi \ldots{} & 5 & Did you ever see such a
sig \ldots{} & 2 & mod\tabularnewline
5 & As three blind mice. \ldots{} & 6 & as three kind mice? \ldots{} & 4
& mod\tabularnewline
6 & End \ldots{} & & & 1 & del\tabularnewline
\bottomrule
\end{longtable}

To gather changes in that manner the first task is to \textbf{acquire
all the documents} that describe the status or evolution of a particular
set of Standing Orders. That step incorporated finding contact persons
within the parliaments and checking for completeness and consistency of
the provided `historical' documents.

While intuitively one might think about Standing Orders as fully written
out, explicit documents most of the time this is only but a little part
of the story. While so called \textbf{consolidated versions} exist, most
of the time one needs a consolidated version and all the amendments
(short, technical instructions of how to transform Standing Orders in
place to a new set of Standing Orders) made to that version over time to
know which set of rules was in place at a certain point in time. To
apply the basic idea all amendments had to be transformed into
consolidated versions.

Documents were provided in differing form and in differing shades of
quality. There might be sheets of paper, Books, Word-documents, machine
readable PDFs or scans. All those various types were first transformed
to Word-documents and later on freed of transformation errors and
artifacts in a \textbf{cleaning} step.

After cleaning and consolidation \textbf{documents were restructured} in
such a way that each sub-paragraph corresponded to one line in a plain
text file. Furthermore, lines without relevant content such as headlines
or notes were marked by \texttt{\#§\#}. The restructuring made it easy
for the documents to be read in by the coding programs used in the
following steps.

For comparing Standing Orders effectively we made use of two types of
programs: First there are programs specialized in presenting the
comparison of documents to humans. Second, there are programs that are
less accessible by humans but more standardized and therefore better
suited to serve as helpers for computer programs. While we found good
companions in the first category -- e.g.~UltraCompare, the Notepad++
Compare Plugin, DiffDoc, WinMerge, \ldots{} see:
\url{https://en.wikipedia.org/wiki/Comparison_of_file_comparison_tools},
for a list) -- we did not find any tool that suited our needs in the
second category -- i.e.~indicating line modifications and measuring
differences.

Therefore we wrote our \textbf{own software} that helped with comparing
texts, assigning change types and measuring differences. Three programs
were written: The frist for comparing documents, \textbf{gathering
links} between sub-paragraph from one version to the other, assigning
\textbf{change types} an measuring change; the second for coding changes
between documents as \textbf{minority or majority friendly}; the third
for coding sub-paragraphs into \textbf{categories capturing the type of
regulation}.

The data gathered with help of the programs than was merged into one
\textbf{database} with three tables - meta information on the Standing
Orders (\emph{texts}), the text of the Standing Orders and accompanied
data (\emph{textlines}) and how sub-paragraphs from one version are
linked those of another version (\emph{textlines}). Thereafter the
information were \textbf{checked for errors}. After elimanting all
errors the raw information from the database was then
\textbf{aggregated} to various formats.

\newpage

\section{Document Transformation}\label{document-transformation}

The original Standing Orders documents gathered often were books or came
as scans in PDF format. To be able to further work with them the text
had to be brought into computer readable format. This first digital
format of the Standing Orders text was Word. Although, the texts later
on were further broken down into plain text, Word is a good intermediate
format since everyone is familiar with it and it is able to emulate the
layout of the original document which eases comparisons between original
and digital version.

\begin{center}
\includegraphics[height=0.7\textheight]{fig/fig1.png}
\end{center}

\emph{Figure 1: 1952 German Bundestag Standing Orders word-document}

\newpage

\section{Document Consolidation}\label{document-consolidation}

In order to construct a data base consisting of all parliamentary
standing orders that were in force at a specific point in time
consolidated versions of the standing orders were needed. Consolidated
versions are complete versions with all changes that had occurred at a
specific date included in the body of the text.

However, there are often only few consolidated versions provided by
national parliaments. Changes to the standing orders are most of the
time published as amendments and only once in a while a full version of
the current text is issued. As a consequence, consolidated versions of
the standing orders for every date of change had to be constructed by
inserting manually the changes into the previous complete version.

\begin{center}
\includegraphics[height=0.7\textheight]{fig/fig1a.png}
\end{center}

\emph{Figure 2: List of ammendments to the 1950 SO UK, page 497}

\newpage

\section{Document Cleaning}\label{document-cleaning}

The aim of this procedure was to identify and correct errors that had
occurred through the process of converting PDF-documents into
Word-documents. Another purpose was to put the cleaned versions of the
parliamentary standing orders in a standardized format while maintaining
the original structure of paragraphs. First of all, the oldest
consolidated version of Standing Orders was completely read through and
corrected manually. Typos, unnecessary line breaks, signs not belonging
to the text and everything going beyond the actual text of the standing
orders was deleted.

Next, a header containing information about the version such as the
dates of acceptance, promulgation and enactment of the version of the
standing orders was inserted. The version cleaned in this manner served
as reference version. In the following step the subsequent consolidated
version was compared to the reference version using the software DiffDoc
(in a later state of the cleaning process the software UltraCompare was
used instead). The software made it possible to easily identify
identical parts of the two versions that only contained few mistakes to
be corrected as well as parts that had been changed. The latter were
read completely and handled like the first version. After the cleaning
of the second version it served as new reference version for the
subsequent consolidated version of the standing orders. Alongside
cleaning the text, it was made sure that each sub-paragraph, headline or
other structuring element of the standing orders was given a single
line. Each element was separated by a line break; no element was allowed
to spread more than one line. The steps were repeated for all
consolidated versions. Throughout the procedure the PDF-versions of the
standing orders were considered in case of uncertainty regarding
cleaning decisions. In a last step the lines (representing text
elements) that were of non-relevant content (e.g.~headlines) were marked
by adding a special string at the start of the line (`\#§\#') to allow
the computer to automatically dismiss these lines later on.

\begin{center}
\includegraphics[width=\textwidth]{fig/fig2.png}
\end{center}

\emph{Figure 3: 1952 SO Germany plain-text version}

\begin{center}
\includegraphics[width=\textwidth]{fig/fig3a.png}
\end{center}

\emph{Figure 4: Comparing Standing Orders versions with Ultra Compare}

\newpage

\section{Linkage}\label{linkage}

\subsection{Description}\label{description}

Having generated complete, cleaned versions of the standing orders, the
next step was to further prepare the data so that the content of the
standing orders could be coded in an efficient manner and to get
information about what had or had not changed from one version to the
next. For this purpose changes in the standing orders between versions
were linked. This means that for each line of text (each relevant
sub-paragraph) it was recorded whether or not the line was deleted in
the version to come, got inserted in the current version, got changed or
simply stayed the same. The coding was done semi-automatically by first
letting an algorithm developed by the project and implemented in R
handle all non-relevant lines as well as those that were not changed.
Thereafter human coders went through all remaining text lines of two
subsequent versions to add there linkage information to the data-set.
For this another program implemented in R helped the coders by making
sure that: all lines were coded; the information was recorded correct
and alongside the text that was linked, coders were given suggestions
for possible matching lines similar to that under consideration. As the
linked files depicted the basis for later analyzes and coding, it was
crucial to differentiate between minor reformulations of paragraphs
(e.g.~mere orthographic reforms) and actual changes. In case of doubt
the supervisors were consulted.

The process of gathering link information between sub-paragraphs of
subsequent standing orders versions allows for distinguishing between
types of change (deletion, insertion, modification and no-change),
measuring its extent more precisely and to later transfer line codes
from one version of the standing orders to another so that all
sub-paragraphs (the selected coding entity) that were identical in two
versions got automatically the same code.

Furthermore, the use of an semi-automatic approach allows to use the
strengths of both computers and humans: computers are good at doing the
same stuff over and over again in the same and predictable way -
e.g.~finding identical lines, computing measures of similarity, saving
data always in the very same format -- while humans on the other hand
have a much better understanding of the content of text, might
understand intentions of the text authors and are more creative and
flexible -- e.g.~finding line pairs that might be not very similar based
on the sequence of characters or the distribution of words but in regard
to the things that are regulated within.

\begin{Shaded}
\begin{Highlighting}[]
\KeywordTok{library}\NormalTok{(stringr)}
\KeywordTok{library}\NormalTok{(diffr)}

\CommentTok{# defining text}
\NormalTok{old <-}\StringTok{ }\KeywordTok{str_split}\NormalTok{(}\StringTok{"Three kind mice, see how they run!}\CharTok{\textbackslash{}n}\StringTok{They all ran after the farmer's wife,}\CharTok{\textbackslash{}n}\StringTok{Who cut off their tails with the carving knife,}\CharTok{\textbackslash{}n}\StringTok{Did you ever see such a thing in your life?}\CharTok{\textbackslash{}n}\StringTok{As three blind mice.}\CharTok{\textbackslash{}n}\StringTok{End"}\NormalTok{, }\StringTok{"}\CharTok{\textbackslash{}n}\StringTok{"}\NormalTok{)}
\NormalTok{new <-}\StringTok{ }\KeywordTok{str_split}\NormalTok{(}\StringTok{"Three blind mice, see how they run!}\CharTok{\textbackslash{}n}\StringTok{They all ran after the farmer's wife,}\CharTok{\textbackslash{}n}\StringTok{they took out some cheese,}\CharTok{\textbackslash{}n}\StringTok{and they cut her a slice,}\CharTok{\textbackslash{}n}\StringTok{Did you ever see such a sight in your life}\CharTok{\textbackslash{}n}\StringTok{as three kind mice?"}\NormalTok{, }\StringTok{"}\CharTok{\textbackslash{}n}\StringTok{"}\NormalTok{)}

\CommentTok{# calculating distances, aligning text, determining change types}
\NormalTok{res <-}\StringTok{ }\KeywordTok{diffr}\NormalTok{(old, new, }\DataTypeTok{dist=}\StringTok{"bow"}\NormalTok{)}

\CommentTok{# distance matrix}
\NormalTok{res$distance_matrix}
\end{Highlighting}
\end{Shaded}

\begin{verbatim}
##      [,1] [,2] [,3] [,4] [,5] [,6]
## [1,]    2   15   10   11   15    7
## [2,]   15    0   13   14   18   12
## [3,]   16   15   14   13   19   13
## [4,]   15   18   15   14    2   14
## [5,]    7   12    9   10   14    4
## [6,]    8    9    6    7   11    5
\end{verbatim}

\begin{Shaded}
\begin{Highlighting}[]
\CommentTok{# resulting alignment and change type}
\NormalTok{res$alignment_df}
\end{Highlighting}
\end{Shaded}

\begin{verbatim}
##   lnr1 lnr2 distance  type
## 1    1    1        2   mod
## 2    2    2        0 equal
## 3    3    4       13   mod
## 7   NA    3        5   ins
## 4    4    5        2   mod
## 5    5    6        4   mod
## 6    6   NA        1   del
\end{verbatim}

\emph{Code Snippet 1: Text comparison with own software written in R}

\begin{center}
\includegraphics[width=\textwidth]{fig/linkage.png}
\end{center}

\emph{Figure 5: R Software with commandline interface for linking parts
of two version of Standing Orders}

\subsection{Training and Quality
Control}\label{training-and-quality-control}

While software was used to help finding matches between versions was
non-trivial. Coders were recruited on the basis that either they were
native speakers or had a very high language proficiency. Good knowledge
about the government system of the countries was a further requirement.
In addition, coders had to undergo training:

\begin{itemize}
\tightlist
\item
  First, coders were informed about the purpose of their task, i.e.:
  Gathering information about changed and unchanged parts of Standing
  Orders to later on build determine the amount of change, as well as
  easing Corpus Coding and Minority/Majority-Coding later on.
\item
  Next, coders were asked to take some time to read the first and the
  last version available to get a feeling for the content, it's
  structure, and the amount of change that happend between first and
  last version.
\item
  Than coders were introduced into DiffDoc and later on UltraCompare to
  view the whole documents side by side if needed.
\item
  Than coders got supervised training on the coding software for one to
  three reforms.
\item
  During most of the coding one of the research assistants was present
  within the same room to answer question and asking questions was
  explicitly encouraged.
\end{itemize}

\newpage

\section{Minority-Majority-Change
Coding}\label{minority-majority-change-coding}

\subsection{Description}\label{description-1}

After having identified which parts of the text were modified, moved
around, deleted or inserted those changes could be coded. For this step
the software drew on the information gathered before in the linkage
stage to confront the coder only with changes instead of all the text.
Coding decisions - pro majority or pro minority - were recorded and
added to the database.

\begin{center}
\includegraphics[width=\textwidth]{fig/minmaj.png}
\end{center}

\emph{Figure 6: R Software with commandline interface for coding
minority/majority proness of reform}

\subsection{Training and Quality
Assurance}\label{training-and-quality-assurance}

Coding the proness of changes to the parliamentary Standing Orders
towards majority or minority is all but trivial. The original standing
orders of national parliaments are usually only published in the
language of the country. Thus, coders were recruited who were either
native speakers or non-native speakers with very high language
proficiency. Good knowledge about the government system of the countries
was a further requirement. In addition, coders had to undergo an
intensive training.

\begin{itemize}
\tightlist
\item
  First, coders were introduced to the concept of parliamentary minority
  and majority and how it differs from Government and other political
  bodies which are often closely connected but might not be the same --
  e.g.~Governments often are dependent on a parliamentary majority that
  is supportive or at least tolerates Government but Government and
  parliamentary majority are not the same thing with agendas and
  personal of their own.
\item
  Next, coders were given one Standing Orders reform with the task to
  try to code on their own. Coders were asked to be careful and if in
  doubt to gather argumentations for why a coding might go in one
  direction or the other.
\item
  After having left the coder with this first reform. All coding
  decissions were discussed.
\item
  This process usually did not have to be repeated but was done so, if
  canonical solutions and individual coding decissions were to much
  apart.
\item
  After this formal training phase coders were asked to code all clear
  neither-nor cases themselves while they should come up with
  suggestions and reasoning for all non clear cases and changes that
  might be either minority or majority friendly.
\item
  For most of the time when a coder was working one of the two research
  assistants was present in the room. Asking questions was strongly and
  always encouraged.
\item
  After the coder did a batch of coding all non-clear cases were
  discussed by the coder and one of the research assistants and decided.
  If need be the project leader entered the discussion as well.
\end{itemize}

\newpage

\section{Corpus Coding}\label{corpus-coding}

\subsection{Description}\label{description-2}

Based on the linked versions of the standing orders the content could be
coded. The intention of the so called corpus coding process was to
assign a single code expressing the content to every legal sub-paragraph
of every version of the standing orders.

The coding scheme comprises 80 different single codes belonging to ten
different categories (law-making, special decision procedures other than
regular law-making, relationship to government, relationship to external
offices/institutions apart from the government, generating publicity,
internal organization of parliament, change and interpretation of the
standing orders, general rules regarding formation and legislative
session/discontinuity, final provisions, miscellaneous (cannot be coded
otherwise)).

Apart from the codes the coding manual encompasses general rules for
coding. As every sub-paragraph got only one code the coders had to
decide which code suites best even if several different codes could be
assigned to a sub-paragraph. These decisions were based on a specific
hierarchy of codes. Rules which concern the interaction of two actors
were attributed to the actor which takes the active part if he has
discretion regarding this action. Regular law-making was considered more
important than other decision procedures if they were treated together
in one sub-paragraph. A further general coding rule was to take the
overall context into account instead of just looking at a specific
regulation.

Like the other coding processes corpus coding was done
semi-automatically with a self-written program implemented in R. Human
coders went through the oldest version of the standing orders and
assigned the appropriate codes from the coding scheme to every text line
to create one fully coded version. The next step was to transfer these
codes to the other versions. As in the linking procedure text lines that
have stayed exactly the same from one version to the following had been
identified and linked, the R program automatically assigned the same
codes to them. So the coders only had to go through the not coded text
lines of the subsequent version of the standing orders (that is the
passages that had been changed between two versions) and code them
manually. Then, the new codes were transferred. The coders proceeded in
this way until all versions of the standing orders were completely coded
with regard to their content.

\subsection{Training and Quality
Assurance}\label{training-and-quality-assurance-1}

The original standing orders of national parliaments are usually only
published in the language of the country. Thus, coders were recruited
who were either native speakers or non-native speakers with very high
language proficiency. Good knowledge about the government system of the
countries was a further requirement. As corpus coding was a very
demanding task, all coders got intensive training.

\begin{itemize}
\tightlist
\item
  First, the coders familiarized themselves with the coding manual, the
  different categories and coding rules.
\item
  Next, the new coders practiced through joint coding with experienced
  coders.
\item
  After this, the coders coded the most recent version of the standing
  orders and compared their results to a master version. Usually the
  most recent versions of the standing orders are also issued in
  English. These versions had been coded by those responsible for the
  project and served as master versions.
\item
  On the condition that coders mastered this task they could start
  coding on their own.
\item
  Throughout the coding process ambiguities and problems were solved in
  joint discussion with the project coordinators.
\item
  If it was necessary, supplementary documents such as constitutions and
  information from the webpages of the national parliaments were
  considered to check coding decisions.
\end{itemize}

\newpage

\section{Appendix}\label{appendix}

\subsection{Manual for Text Cleansing}\label{manual-for-text-cleansing}

\textbf{First steps}

\begin{itemize}
\tightlist
\item
  After Julia assigned you a country, please write your name in the top
  part of the file ``Säuberung\_Geschäftsordnungen.xlsx''
  (Z:/Geschäftsordnungen/Übersichtsdateien/Säuberung\_Geschäftsordnungen.xlsx)
\item
  Complete all dates (Date of Acceptance, Date of Promulgation, Date of
  Enactment) in this file.
\item
  As soon as a version is corrected, please highlight this in the file
  (highlight line in green, put counter from ``1''" to ``0'')
\item
  To make sure that the original consolidated versions remain unchanged,
  please copy (DO NOT DRAG AND DROP, BUT ACTUALLY COPY THEM) all
  consolidated versions to the folder ``gesäuberte Versionen''
  (Z:/Geschäftsordnungen/Daten/CountryName/Gesäuberte Versionen)
\item
  If this folder does not yet exist, just create it yourself
\end{itemize}

\textbf{Procedure}

\begin{itemize}
\tightlist
\item
  First (oldest) consolidated Version

  \begin{itemize}
  \tightlist
  \item
    Read it completely and insert/delete the information listed below
  \item
    Save this cleaned version as a .doc
  \item
    Use this version as a reference for the following version
  \end{itemize}
\item
  Other consolidated versions
\item
  Compare the subsequent consolidated version through DiffDoc with the
  reference version + Parts that are identical except for mistakes:
  correct the mistakes in the newer version + Parts that have been
  changed: read these passages completely and clean it like the first
  version + Save this version

  \begin{itemize}
  \tightlist
  \item
    Use this version as a reference for the subsequent version, and so
    on
  \end{itemize}
\item
  Note for UltraCompare: Converting the PDF documents to Word documents
  resulted in so-called ``Bedingte Trennstriche'' (soft hyphen). These
  are invisible in the Word documents, but UltraCompare recognizes them.
  You can delete these by using the ``Suchen und Ersetzen'' (find and
  replace) function: at the very bottom, select ``Sonderformat''
  (special format) and then check the option ``Bedingter Trennstrich''
  (soft hyphen). Then, you select the ``Replace'' option and replace
  this ``Bedingter Trennstrich'' with nothing (e.g.~not even a space,
  just leave the line blank).
\end{itemize}

\textbf{Information that should be inserted/deleted}

\begin{itemize}
\tightlist
\item
  \textbf{IMPORTANT: PLEASE DO NOT ``CORRECT'' THE TEXT WITH REGARDS TO
  CONTENT -- SOME EXPRESSIONS ETC. MAY SEEM ODD, BUT WE ARRE ONLY
  LOOKING TO ELIMINATE ANY MISTAKES THAT CAME INTO EXISTENCE WHEN THE
  TEXT WAS TRANSFERRED FROM .pdf TO .doc!!!}
\item
  Delete

  \begin{itemize}
  \tightlist
  \item
    Unnecessary line breaks (e.g.~if a sentence is intersected by a line
    break, but the sentence clearly should be in one line)
  \item
    all paragraphs and sub-paragraphs should begin in a new line, and
    there should not be any other line breaks
  \item
    Typos
  \item
    Weird signs which do not belong in the text (\%, \&, \$)
  \item
    Double space characters
  \item
    Headers and footers
  \item
    At the beginning of the text: all the stuff before the first
    headline
  \end{itemize}
\item
  Insert/change

  \begin{itemize}
  \tightlist
  \item
    At the beginning of the document:

    \begin{itemize}
    \tightlist
    \item
      Country
    \item
      Title of the Document
    \item
      Date of Acceptance
    \item
      Date of Promulgation
    \item
      Date of Enactment
    \end{itemize}
  \item
    Example:

    \begin{itemize}
    \tightlist
    \item
      Germany
    \item
      Geschäftsordnung des Bundestages
    \item
      Date of Acceptance: 05.11.1969
    \item
      Date of Promulgation: 10.11.1969
    \item
      Date of Enactment: (if there is no Date of Enactment, do not write
      anything after the colon)
    \end{itemize}
  \item
    Footnotes

    \begin{itemize}
    \tightlist
    \item
      Replace the numbers of the footnotes in the text by
      {[}{[}FN:NUMBER{]}{]}, ``NUMBER' being a place holder for the
      number of the footnote
    \item
      Insert the text of the footnote at the end of the paragraph into a
      new line. This line starts with {[}{[}FT:NUMBER{]}{]} `NUMBER'
      again being a place holder for the number of the footnote
    \item
      Example

      \begin{itemize}
      \tightlist
      \item
        text text1 text\ldots{} --\textgreater{} text text
        {[}{[}FN:1{]}{]} text \ldots{}
      \item
        new line: {[}{[}FT:1{]}{]} text of the footnote
      \end{itemize}
    \item
      Save the document like this
    \item
      In a next step, the footnotes have to be deleted
    \item
      save this document as well
    \end{itemize}
  \end{itemize}
\end{itemize}

\textbf{Saving the new documents}

\begin{itemize}
\tightlist
\item
  Word documents

  \begin{itemize}
  \tightlist
  \item
    Names: Abbreviation of country-Year\_Month\_Date-cons-sauber-mfn/ofn

    \begin{itemize}
    \tightlist
    \item
      Months and days always have to be included (use ``Date of
      Enactment'')
    \item
      ``mfn'' stands for ``with footnotes'', ``ofn'' stands for
      ``without footnotes''
    \item
      Example: ITA-1999\_12\_16-cons-sauber-ofn
    \end{itemize}
  \item
    Save the documents to the following location:
    Geschäftsordnungen/Daten/Länderordner/Gesäuberte Versionen
  \item
    If this folder does not yet exist, create it yourself
  \end{itemize}
\end{itemize}

\textbf{Using Ultra Compare}

\begin{itemize}
\tightlist
\item
  Open UltraCompare
\item
  Option for document comparison becomes visible
\item
  Before loading the documents, do the following

  \begin{itemize}
  \tightlist
  \item
    Tab ``Ansicht'': choose ``Zeilenumbruch ein''; choose
    ``Zeilenbezug-Modus''
  \item
    Tab ``Modus'': choose ``Textmodus''
  \item
    Symbol ``Sitzungseigenschaften'' (this symbol looks like a
    gearwheel)

    \begin{itemize}
    \tightlist
    \item
      Tab ``Vergleichsoptionen'': choose ``Beziehungslinien anzeigen''
      and choose ``Wortumbruch''
    \item
      Tab ``Ignorieren'': choose ``leere Zeilen ignorieren'' and choose
      ``Weißraum ignorieren''
    \item
      Tab ``Spaltenbereiche vergleichen/ignorieren'': in the field
      ``Ignorieren'', type in ``1-4''
    \item
      this leads to the first four characters being ignored, which is
      very helpful if the numbering changed. However, you now have to
      manually check if there are spelling mistakes or the like in the
      first four characters!
    \end{itemize}
  \end{itemize}
\item
  Load the documents
\item
  Start comparison through hitting the green arrow
\item
  Comparison

  \begin{itemize}
  \tightlist
  \item
    Black lettering: no change
  \item
    changes

    \begin{itemize}
    \tightlist
    \item
      \textless{}!: line deleted
    \item
      !\textgreater{}: line added
    \item
      *: line changed
    \item
      Red: letter was changed
    \item
      Blue: letter remained the same
    \end{itemize}
  \end{itemize}
\item
  Corrections can be done directly in UltraCompare, it works like a
  standard text editor
\item
  To save the document, select the tab ``Datei''

  \begin{itemize}
  \tightlist
  \item
    select ``Speichern Unter''
  \item
    select correct place and name for saving
  \item
    hit ``Speichern''
  \end{itemize}
\end{itemize}

\newpage

\subsection{Manual for Linking Standing Orders
versions}\label{manual-for-linking-standing-orders-versions}

\textbf{Before linking the changes in R}, the following steps have to be
taken

\begin{enumerate}
\def\labelenumi{\arabic{enumi})}
\tightlist
\item
  Save all cleaned versions as txt-files in ``Geschäftsordnungen/Coding
  Changes/CountryFolder/TXT''
\item
  Make sure the naming of the TXT-files complies to the following
  scheme:

  \begin{enumerate}
  \def\labelenumii{\alph{enumii}.}
  \tightlist
  \item
    Country abbreviation
  \item
    minus
  \item
    year (4 digits)
  \item
    underscore
  \item
    month (2 digits)
  \item
    underscore
  \item
    day (2 digits)
  \item
    if and only if there are two versions for the same date indicate
    them with:
  \item
    underscore
  \item
    letter (A, B, C, \ldots{} for the first, second, \ldots{} version on
    that date, make sure to use CAPITAL Letters)
  \item
    Examples: ``SWE-2002\_09\_23.txt'', ``SWE-2003\_01\_01\_A.txt'',
    ``SWE-2003\_01\_01\_B.txt'', \ldots{}
  \end{enumerate}
\item
  Put \#§\# (hash paragraph hash space) before all irrelevant lines
  (e.g.~headlines, name of standing orders, date of going into force)
\end{enumerate}

\textbf{The intuition behind linking the changes} is as follows:

\begin{itemize}
\tightlist
\item
  The R-function recognizes automatically when two sub-paragraphs are
  absolutely identical - if this is the case, they are automatically
  coded as 100 (\% identical)
\item
  All other sub-paragraphs are presented by the R-function for manual
  linking (this is even the case if there is just a space too much in
  either one of the sub-paragraphs)
\item
  Whether or not two sub-paragraphs belong together can be assessed best
  when comparing the versions one is linking in UltraCompare
\end{itemize}

\textbf{Linking in R}

\begin{enumerate}
\def\labelenumi{\arabic{enumi})}
\tightlist
\item
  Open R
\item
  Open Notepad
\item
  Execute one of the following functions

  \begin{enumerate}
  \def\labelenumii{\alph{enumii}.}
  \tightlist
  \item
    Begin linking: ``Coding Changes''
  \item
    Continue linking: ``Recode Changes''
  \end{enumerate}
\item
  Comply with the prompts of the function

  \begin{enumerate}
  \def\labelenumii{\alph{enumii}.}
  \tightlist
  \item
    Type in your name
  \item
    Select the original documents (Geschäftsordnungen/Coding
    Changes/CountryFolder/TXT/txt-file)
  \item
    Press enter
  \end{enumerate}
\item
  Link non-identical sub-paragraphs (absolutely identical ones are
  linked automatically)

  \begin{itemize}
  \tightlist
  \item
    0: deletion/insertion of a sub-paragraph
  \item
    1: change of a sub-paragraph
  \item
    100: sub-paragraph is 100\% identical
  \item
    -99: line is irrelevant (this option should have become unnecessary
    through \#§\# )
  \item
    -7: show next match
  \item
    321: correct line number can be inserted directly (useful if the
    proposed lines do not include the correct line)
  \item
    666: loop starts again
  \item
    987: not sure how sub-paragraph should be coded
  \item
    951: end coding --\textgreater{} every subsequent sub-paragraph will
    be coded with 987 (if you choose to use this option, write down the
    last ``real'' 987 to allow you to continue coding at the right point
    the next time)
  \end{itemize}
\end{enumerate}

\textbf{FAQ}

\textbf{Linked as 100\% identical (i.e.~100)}

\begin{itemize}
\tightlist
\item
  When original document says ``in the article number 2'' and the new
  version says ``in the second article'' (and vice versa)
\item
  When original document says ``The commissions are: (and then a list of
  commissions)'', and the new version says ``The commissions are the
  following: (and then a list of commissions)''
\item
  When original document says ``The Parliament'', and the new version
  says ``The National Parliament'', as long as it is clear they refer
  the same institution
\item
  When the difference between original and new version is:

  \begin{itemize}
  \tightlist
  \item
    A comma
  \item
    A space in the middle or end of the sentence
  \item
    A capital letter
  \end{itemize}
\item
  When the same article refers to a paragraph that has a different
  number (``according to article 1'' in the original, and ``according to
  the article 2'' in the new version), but the content of them is the
  same
\end{itemize}

\textbf{Linked as change (i.e.~1)}

\begin{itemize}
\tightlist
\item
  When the same article refers to a paragraph that has a different
  number (``according to article 1'' in the original, and ``according to
  the article 2'' in the new version), but the content of these is not
  the same
\item
  When the same article refers to the same paragraph that has the same
  number (``according to article 1'' in the original, and ``according to
  article 1'' in the new version), but the content of these (article 1)
  is not the same
\item
  When the article has changed
\item
  When part of the original article was deleted (original reads ``There
  are 15 deputies in each commission as minimum, and they cannot exceed
  30'', and the new version reads ``There are 15 deputies in each
  commission as minimum'')
\item
  When the original document says ``according to the President'' and the
  new document says ``according to the President of the Parliament''
  when the first one could also refer the President of a special
  commission
\end{itemize}

\textbf{ATTENTION: If in doubt, ALWAYS ask us for help! All of our
analyses are based on these linked files!}

\newpage

\subsection{Coding Scheme for Corpus
Coding}\label{coding-scheme-for-corpus-coding}

\subsubsection{Basic Intuition:}\label{basic-intuition}

Each and every code is exclusive, meaning that one sub-paragraph needs
to have one code but one code only. For some codes there are notes on
how to decide between multiple codes which may seem appropriate.
Sometimes even the coding rules and additional notes will not help to
decide between codes. In this case please let us know. Every decision
accompanied by doubt should be documented.

\subsubsection{Further rules of the
game:}\label{further-rules-of-the-game}

\begin{itemize}
\item
  Often sub-paragraphs can be coded differently, depending on whether or
  not one takes into consideration the overall context of the rule or
  the more specific regulation. If in doubt, code based on the overall
  context. Example: §14 GOBT: president grants vacation time → coded as
  rights and obligations of individual members of parliament if one
  takes into account the general context (652) and not as responsibility
  of the president (6212).
\item
  Rules which concern the interaction of two actors are attributed to
  the actor which takes the active part if he has discretion regarding
  this action. Example: §62 (2) GOBT: The plenary can request report of
  committee → coded as recall through the plenary (124) and not as
  report of committee to the plenary (134).
\item
  The right of those initiating a bill or law to be present at the
  committee meetings is coded as general right of individual members of
  parliament (652).
\end{itemize}

\subsubsection{Scheme}\label{scheme}

\textbf{(1) Law Making}

~

\emph{Note: SPs that refer to both the plenary sessions and committees
are coded as 12x; SPs dealing with both law-making and special decision
procedures are coded as 1xx.}

\begin{itemize}
\tightlist
\item
  \textbf{11 Bills and Motions}

  \begin{itemize}
  \tightlist
  \item
    111 types of bills and motions; printing and distribution of bills
    and motions to MPs
  \item
    112 right to initiate bills and motions
  \item
    113 restrictions and deadlines (if not assignable to more specific
    category, e.g.~code 121; 32; 134)
  \item
    114 legislative planning (concerns the whole term- general schedule)
  \end{itemize}
\item
  \textbf{12 Treatment of bills and motions in the plenary} \emph{(Note:
  SPs including all stages of the treatment of bills and motions are
  coded as votes in the plenary (123); SPs which determine the subject
  of debate and vote are coded as subject of vote (123).)}

  \begin{itemize}
  \tightlist
  \item
    121 debate in the plenary
  \item
    122 right of amendment in the plenary
  \item
    123 subject of vote, rules of vote (including quorum), voting
    technology in the plenary
  \item
    124 the plenary as Committee of the Whole House \emph{(Note: SPs
    referring to both committees and Committee of the Whole House are
    coded as committees (not 124 but 13x).)}
  \item
    125 referral to committee, withdrawal from committee
  \end{itemize}
\item
  \textbf{13 Treatment of bills and motions in committee} \emph{(Note:
  SPs including all stages of the treatment of bills and motions in
  committee are coded as votes in committee (133); SPs which determine
  both the subject of debate and the subject of vote are coded as
  subject of vote in committee (133).)}

  \begin{itemize}
  \tightlist
  \item
    131 debate in committee (including hearing)
  \item
    132 amendment rights in committee
  \item
    133 subject of vote, rules of vote (including quorum), voting
    technology in committee
  \item
    134 report to the plenary
  \end{itemize}
\end{itemize}

\textbf{(2) Special Decision Procedures other than Regular Law-Making}

~

\emph{Note: SPs which concern multiple special decision procedures apart
from regular law-making are coded as follows: highest priority is given
to constitutional matters, second highest priority is given to financial
laws and budgeting, third highest priority is given to EU policy and
fourth highest priority is given toforeign policy.}

\begin{itemize}
\tightlist
\item
  \textbf{21 constitutional change and amendment}
\item
  \textbf{22 financial laws} (money bills) and budgeting
\item
  \textbf{23 foreign policy} (e.g.~approval of law of nations,
  declaration of war \emph{Note: If foreign policy and EU is treated
  together, the SP is coded as EU (241, 242, 243 or 244).})
\item
  \textbf{24 EU} \emph{(Note: If foreign policy and EU is treated
  together, the SP is coded as EU (241, 242, 243 or 244))}

  \begin{itemize}
  \tightlist
  \item
    241 treatment of EU-bills and motions
  \item
    242 EU-committee: election and resignation
  \item
    243 instructions to the government concerning EU decisions
  \item
    244 further rights of participation in EU matters (e.g.~debates
    about EU topics not based on EU bills and motions, reaction to
    violations of subsidiary principle)
  \end{itemize}
\item
  \textbf{25 general rules on elections in parliament} (if not coded as
  election of government (31), or election of specific officials (411;
  421; 441; 6211; 6221; 632))
\item
  \textbf{26 further special decision procedures} leading to a decision,
  e.g.~resolution, or leading to a decree/act/bylaw (not mere debate or
  question time) but cannot be coded as regular law-making nor special
  decision procedures (21-24)
\item
  \textbf{27 procedures concerning laws that are hierarchically situated
  between regular laws and constitutional laws} (above regular laws;
  e.g.~organic laws in Spain)
\item
  \textbf{28 emergency legislation}
\item
  \textbf{29 relationship to sub-national level} (law-making, rights of
  participation of sub-national level)
\end{itemize}

\textbf{(3) Relationship to Government}

~

\emph{Note: If vote of no confidence and vote of confidence is treated
together, the SP is coded as vote of no confidence (32).}

\begin{itemize}
\tightlist
\item
  \textbf{31 election of government / mandatory investiture vote; entry
  into office}
\item
  \textbf{32 vote of no confidence / government resignation}
\item
  \textbf{33 vote of confidence}
\item
  \textbf{34 instructions to government, involvement of members of
  government in parliamentary activities} (rights to compel witnesses
  {[}usually right of parliament against members of government{]}, right
  to speak {[}usually members of government's right{]}, request of
  information about state of execution of decisions of parliament)
\end{itemize}

\textbf{(4) Relationship to External Offices/Institutions apart from the
Government}

\begin{itemize}
\tightlist
\item
  \textbf{41 parliamentary support bodies (e.g.~general accounting
  office, ombudsman,\ldots{})}

  \begin{itemize}
  \tightlist
  \item
    411 election and resignation
  \item
    412 competences and resources of external offices/institutions;
    relations to parliament (e.g.~reports, questions, \ldots{})
  \end{itemize}
\item
  \textbf{42 head of state}

  \begin{itemize}
  \tightlist
  \item
    421 election and resignation
  \item
    422 relation to parliament (if not coded as law-making (141, 144))
  \end{itemize}
\item
  \textbf{43 second chamber (if not coded as law-making (142))}
\item
  \textbf{44 constitutional courts}

  \begin{itemize}
  \tightlist
  \item
    441 election and resignation
  \item
    442 relation to parliament (if not coded as law-making (145))
  \end{itemize}
\item
  \textbf{45 other external offices}
\end{itemize}

\textbf{(5) Generating Publicity}

\begin{itemize}
\tightlist
\item
  \textbf{51 general rules regarding debate} (e.g.~time allotted for
  speaking, proportional representation of parties during debate,
  closure of debate)
\item
  \textbf{52 debates outside of law-making} (e.g.~topical hours
  \ldots{})
\item
  \textbf{53 question rights}
\item
  \textbf{54 petitions and petition committee}
\item
  \textbf{55 relationship to media and citizens} (e.g.~parliamentary TV,
  accreditation of journalists, publicity of meetings, admissibility of
  visitors); regulation of matters of confidentiality
\item
  \textbf{56 protocols and parliamentary documents; forwarding of
  documents and decisions to other bodies}
\end{itemize}

\textbf{6 Internal Organization of Parliament}

\begin{itemize}
\tightlist
\item
  \textbf{61 plenary}

  \begin{itemize}
  \tightlist
  \item
    611 agenda setting and removal of items from the agenda (general
    rules which are not specifically regulated under 114)
  \item
    612 chairing of meetings and measures to uphold order
  \item
    613 sitting times \emph{(Note: When members are to be present inside
    the parliament)}
  \end{itemize}
\item
  \textbf{62 parliamentary presiding bodies}

  \begin{itemize}
  \tightlist
  \item
    621 president of parliament, vice presidents, secretaries and clerks

    \begin{itemize}
    \tightlist
    \item
      6211 election, resignation and internal decision rules
    \item
      6212 responsibilities \emph{(Note: if not coded as more specific
      category (e.g.~612), Try to code in regard to the topic at first -
      6212 only when no code corresponds)}
    \end{itemize}
  \item
    622 council of elders or similar coordination body \emph{(Note: The
    council of elders can be distinguished from the Presidency of
    Parliament (621) insofar as representatives of the parliamentary
    party groups are explicitly included.)}

    \begin{itemize}
    \tightlist
    \item
      6221 composition, election, resignation, internal decision rules
    \item
      6222 responsibilities (if not coded as more specific category
      (e.g.~612))
    \end{itemize}
  \end{itemize}
\item
  \textbf{63 committees} (if not coded as more specific category
  (e.g.~13; 24; 54; 55; 72))

  \begin{itemize}
  \tightlist
  \item
    631 general regulations regarding types of committees
  \item
    632 membership and committee jurisdiction (area of influence-control
    .g. finance, economy, agriculture\ldots{})
  \item
    633 formal organizational units of committee \emph{(Note: e.g.~chair
    of committee, sub-committees, staff; This is about the appointment
    and election of the organizational units within committees and NOT
    about their responsibilities.)}
  \item
    634 agenda and procedures (details on how decisions are taken)
    within committees (if not coded as law-making (13))
  \item
    635 relations to other bodies

    \begin{itemize}
    \tightlist
    \item
      6351 relation to plenary (if not coded as 124; 134; 34)
    \item
      6352 relation to other committees
    \end{itemize}
  \item
    636 investigative competencies of regular committees (NOT committees
    of inquiry (637))

    \begin{itemize}
    \tightlist
    \item
      637 committee of inquiry
    \item
      638 enquête commission
    \end{itemize}
  \item
    639 other special committees which are not explicitly referenced in
    this coding manual \emph{(Note: e.g.~oversight committees in
    Switzerland; Otherwise referenced are: EU-committee (242); committee
    of inquiry (637); petition committee (54); standing order committee
    (usually 72); council of elders or similar coordination body (622).
    Exception: committees which deal exclusively with the confirmation
    of the elections of members of parliament are coded as 651.)}
  \end{itemize}
\item
  \textbf{64 parliamentary party groups}

  \begin{itemize}
  \tightlist
  \item
    641 formation of parliamentary party groups
  \item
    642 rights and obligations of parliamentary party groups (if not
    coded more specifically as e.g.~112; 51; 52; 53)
  \item
    643 financial and staff resources
  \end{itemize}
\item
  \textbf{65 individual members of parliament}

  \begin{itemize}
  \tightlist
  \item
    651 election, entry into office, resignation, incompatibilities,
    legal status, immunity, indemnity
  \item
    652 rights and obligations of individual members of parliament (if
    not coded more specifically as e.g.~112; 51; 52; 53)
  \item
    653 salary, financial and staff resources
  \end{itemize}
\item
  \textbf{66 opposition}
\item
  \textbf{67 special bodies for emergency situations}
\item
  \textbf{68 parliamentary administration}
\end{itemize}

\textbf{7 Change and Interpretation of the Standing Orders}

\begin{itemize}
\tightlist
\item
  \textbf{71 rules regarding changing the standing orders}
\item
  \textbf{72 rules regarding interpretation of and deviation from
  standing orders}
\item
  \textbf{73 debate about standing orders and motions regarding the
  standing orders}
\end{itemize}

\textbf{8 General Rules Regarding Formation and Legislative Session;
Discontinuity}

\textbf{9 Final Provisions}

\textbf{10 Miscellaneous} (cannot be coded otherwise)

\textbf{999 Footnotes and Titles Without Relevant Content}

\end{document}
